%% KO_template.tex
%% V2015
%% by Jan Dvořák
%% based on
%% bare_conf.tex
%% V1.3
%% 2007/01/11
%% by Michael Shell

%%*************************************************************************
%% Legal Notice:
%% This code is offered as-is without any warranty either expressed or
%% implied; without even the implied warranty of MERCHANTABILITY or
%% FITNESS FOR A PARTICULAR PURPOSE! 
%% User assumes all risk.
%% In no event shall IEEE or any contributor to this code be liable for
%% any damages or losses, including, but not limited to, incidental,
%% consequential, or any other damages, resulting from the use or misuse
%% of any information contained here.
%%
%% All comments are the opinions of their respective authors and are not
%% necessarily endorsed by the IEEE.
%%
%% This work is distributed under the LaTeX Project Public License (LPPL)
%% ( http://www.latex-project.org/ ) version 1.3, and may be freely used,
%% distributed and modified. A copy of the LPPL, version 1.3, is included
%% in the base LaTeX documentation of all distributions of LaTeX released
%% 2003/12/01 or later.
%% Retain all contribution notices and credits.
%% ** Modified files should be clearly indicated as such, including  **
%% ** renaming them and changing author support contact information. **
%%
%% File list of work: IEEEtran.cls, IEEEtran_HOWTO.pdf, bare_adv.tex,
%%                    bare_conf.tex, bare_jrnl.tex, bare_jrnl_compsoc.tex
%%*************************************************************************


%%!!!!!!!!!!!!!!!!!!!!!!!!!!!!!!!!!!!!!!!!!!!!!!!!!!!!!!!!!!!!!!!!!!!!!!!!!
%%
%% Remove all itemize environments and replace them by your own text! 
%%
%%!!!!!!!!!!!!!!!!!!!!!!!!!!!!!!!!!!!!!!!!!!!!!!!!!!!!!!!!!!!!!!!!!!!!!!!!!



\documentclass[onecolumn, conference]{IEEEtran}

% Uncomment for seminar work in czech language
%\usepackage[czech]{babel}

%Uncomment for setting up your encoding (according to *.tex encoding you use)
%\usepackage[latin2]{inputenc} %for windows users
%\usepackage[utf8]{inputenc} %for linux users

\newcommand{\firstDeadline}{Deadline: 09.03.\,\the\year}
\newcommand{\secondDeadline}{Deadline: 23.03.\,\the\year}
\newcommand{\fourthDeadline}{Deadline: 04.05.\,\the\year}
\newcommand{\range}[2][0]{Range: #1 to #2 page} 
\newcommand{\conciseItem}{\itemsep1pt \parskip0pt \parsep0pt}

\begin{document}
%
% paper title
% can use linebreaks \\ within to get better formatting as desired
\title{Paper Title}


% author names and affiliationsb
% use a multiple column layout for up to three different
% affiliations
\author{\IEEEauthorblockN{Authors Name}
\IEEEauthorblockA{
Day and time of your parallel class\\
\textit{Study programme}\\
\textit{E-mail address}}
}

% make the title area
\maketitle
\begin{abstract}
\bf
This electronic document is a ``live'' template and already defines the components of your paper [title, text, heads, etc.] in its style sheet.
\end{abstract}


\section{Assignment}
\subsection{Problem Statement}
\begin{itemize}
	\conciseItem
	\item The concise and unambiguous description of your original problem
	\item Do not forget: If there are more seminar works dealing with the same topic, the points will be proportionately reduced.
	\item \range[0.3]{1}
	\item \firstDeadline~(dd.mm\,yyyy)
\end{itemize}
\subsection{Problem Categorization}
\begin{itemize}
	\conciseItem
	\item The categorization of the problem and subproblems (if there are any)
	\item \range[0.2]{0.5}
	\item \firstDeadline
\end{itemize}

\section{Related works}
\begin{itemize}
	\conciseItem
	\item Continuous text with references to the literature (at least 3 relevant scientific articles) including correct citations. For example papers\cite{RW:Eason, RW:Clerk, RW:Jacobs, RW:Elissa, RW:Nicole}.
	\item \range[0.3]{0.5}
	\item \secondDeadline
\end{itemize}


\section{Problem solution}
\subsection{Design}
\begin{itemize}
	\conciseItem
	\item The idea and the design of your solution. Reasons why you choose your method.
	\item \range[0.2]{0.5}
	\item \secondDeadline
\end{itemize}
\subsection{Implementation}
\begin{itemize}
	\conciseItem
	\item Implementation details of your algorithm. ILP problem formulation/description of heuristics/important non-trivial data structures etc.
	\item \range[0.5]{1}
	\item \fourthDeadline
\end{itemize}
\section{Experiments}
\subsection{Benchmark Settings}
\begin{itemize}
	\conciseItem
	\item Description of benchmark instances, testing methodology, environment (CPU, Memory, OS, etc.)
	\item \range[0.3]{0.5}
	\item \fourthDeadline
\end{itemize}
\subsection{Results}
\begin{itemize}
	\conciseItem
	\item Presentation of the results(tables, charts etc.) of your quality, performance/execution time, memory complexity experiments and its description.
	\item \range[0.7]{1.2}s
	\item \fourthDeadline
\end{itemize}
\subsection{Discussion}
\begin{itemize}
	\conciseItem
	\item Discussion of the results. Comparison of the results with the state-of-the-art. Description of expected and unexpected phenomena in the results etc.
	\item \range[0.2]{0.5}
	\item \fourthDeadline
\end{itemize}

\section{Conclusion}
\begin{itemize}
	\conciseItem
	\item Conclusion of the seminar work.
	\item \range[0.2]{0.3}
	\item \fourthDeadline
\end{itemize}



% references section

% can use a bibliography generated by BibTeX as a .bbl file
% BibTeX documentation can be easily obtained at:
% http://www.ctan.org/tex-archive/biblio/bibtex/contrib/doc/
% The IEEEtran BibTeX style support page is at:
% http://www.michaelshell.org/tex/ieeetran/bibtex/
%\bibliographystyle{IEEEtran}
% argument is your BibTeX string definitions and bibliography database(s)
%\bibliography{IEEEabrv,../bib/paper}
%
% <OR> manually copy in the resultant .bbl file
% set second argument of \begin to the number of references
% (used to reserve space for the reference number labels box)
\begin{thebibliography}{1}

\bibitem{RW:Eason}
G. Eason, B. Noble, and I.N. Sneddon, \emph{“On certain integrals of Lipschitz-Hankel type involving products of Bessel functions,“} Phil. Trans. Roy. Soc. London, vol. A247, pp. 529-551, April 1955. (references)

\bibitem{RW:Clerk}
J. Clerk Maxwell, \emph{"A Treatise on Electricity and Magnetism,"} 3rd ed., vol. 2. Oxford: Clarendon, 1892, pp.68-73.

\bibitem{RW:Jacobs}
I.S. Jacobs and C.P. Bean, \emph{“Fine particles, thin films and exchange anisotropy,”} in Magnetism, vol. III, G.T. Rado and H. Suhl, Eds. New York: Academic, 1963, pp. 271-350.

\bibitem{RW:Elissa}
K. Elissa, \emph{“Title of paper if known,”} unpublished.

\bibitem{RW:Nicole}
R. Nicole, \emph{“Title of paper with only first word capitalized,”} J. Name Stand. Abbrev., in press.

\end{thebibliography}
\end{document}